\documentclass[12pt,twoside,a4paper]{article} 

\usepackage{color,amssymb,amsmath,mathtools} 
\usepackage{fullpage,caption, listings,clrscode,placeins} 
 
 % Metadata info
\newcommand{\mytitle}{Project 1} 
\newcommand{\mydate}{\today} 
\newcommand{\myauthors}{Shimian Zhang}

% Setting Hyperref parameters
\usepackage[
	bookmarks,
	bookmarksnumbered,
	pdfpagemode={UseOutlines},
	plainpages=false,
	pdfpagelabels=true,
	pdfauthor={\myauthors},
	pdftitle={\mytitle},
	pagebackref=true,
	pdftex,
	colorlinks=true,
	linkcolor=red,
	urlcolor={blue}, 
	pagebackref=true]
	{hyperref}

\title{\mytitle}
\author{\myauthors}
\date{\mydate}

% Some useful commands from CVPR
\usepackage{xspace}
\makeatletter
\DeclareRobustCommand\onedot{\futurelet\@let@token\@onedot}
\def\@onedot{\ifx\@let@token.\else.\null\fi\xspace}
\def\eg{\emph{e.g}\onedot} \def\Eg{\emph{E.g}\onedot}
\def\ie{\emph{i.e}\onedot} \def\Ie{\emph{I.e}\onedot}
\def\cf{\emph{c.f}\onedot} \def\Cf{\emph{C.f}\onedot}
\def\etc{\emph{etc}\onedot} \def\vs{\emph{vs}\onedot}
\def\wrt{w.r.t\onedot} \def\dof{d.o.f\onedot}
\def\etal{\emph{et~al}\onedot}
\makeatother


\pagestyle{empty}
\usepackage{fancyref,fancyhdr}
%\usepackage[hmarginratio=1:1, top=2.0cm, bottom=5.0cm, left=1cm, right=1cm]{geometry}
\setlength{\headheight}{14pt}
\setlength{\headsep}{15pt}
\setlength{\footskip}{50pt}
\pagestyle{fancy}
\fancyhf{}
\fancyhf[HLE,HRO]{\footnotesize{\myauthors}}
\fancyhf[HLO,HRE]{\footnotesize{\mydate}}
\fancyhf[FLO,FRE]{\footnotesize{\mytitle}} 
\fancyhf[FLE,FRO]{\thepage }


\usepackage[pdftex]{graphicx}
\DeclareGraphicsExtensions{.pdf,.png,.jpg,.eps}

% 
\usepackage[numbers, sort&compress]{natbib}

\usepackage[senames,dvipsnames,svgnames,table]{xcolor}

\newenvironment{tightitemize} % Defines the tightitemize environment which modifies the itemize environment to be more compact
{\vspace{-\topsep}\begin{itemize}\itemsep1pt \parskip0pt \parsep0pt}
{\end{itemize}\vspace{-\topsep}} 

% If you want to write proofs
\newtheorem{claim}{Claim}[section]
\newtheorem{lemma}{Lemma}[section]

% Some useful packages (look at booktabs for good looking tables)
\usepackage{subcaption,booktabs,placeins}


\begin{document}
\maketitle


\begin{abstract}
Brief Abstract describing the project.
\end{abstract}
\vspace{1ex}
\tableofcontents
% Break until the next page
\vfill
\pagebreak

\section{Introduction}
Write what you are planning on doing, a bit of information about each one, any troubles you had with the project.  Make sure to cite anything.  The following are some useful rules for publishing in CVPR. 


%-------------------------------------------------------------------------
\subsection{Type-style and fonts}

Wherever Times is specified, Times Roman may also be used. If neither is
available on your word processor, please use the font closest in
appearance to Times to which you have access.

%-------------------------------------------------------------------------
\subsection{Footnotes}

Please use footnotes\footnote {This is what a footnote looks like.  It
often distracts the reader from the main flow of the argument.} sparingly.
Indeed, try to avoid footnotes altogether and include necessary peripheral
observations in
the text (within parentheses, if you prefer, as in this sentence).  If you
wish to use a footnote, place it at the bottom of the column on the page on
which it is referenced. Use Times 8-point type, single-spaced.

%-------------------------------------------------------------------------
\subsection{References}

List and number all bibliographical references in 9-point Times,
single-spaced, at the end of your paper. When referenced in the text,
enclose the citation number in square brackets, for
example~\cite{ChrisPRML}.  Where appropriate, include the name(s) of
editors of referenced books.


%-------------------------------------------------------------------------
\subsection{Illustrations, graphs, and photographs}

All graphics should be centered.  Please ensure that any point you wish to
make is resolvable in a printed copy of the paper.  Resize fonts in figures
to match the font in the body text, and choose line widths which render
effectively in print. 

\begin{table}[!ht]
\centering
\begin{tabular}{llr}  
\toprule
\multicolumn{2}{c}{Item} \\
\cmidrule(r){1-2}
Animal    & Description & Price (\$) \\
\midrule
Gnat      & per gram    & 13.65      \\
          &    each     & 0.01       \\
Gnu       & stuffed     & 92.50      \\
Emu       & stuffed     & 33.33      \\
Armadillo & frozen      & 8.99       \\
\bottomrule
\end{tabular}
\caption{Results.   Ours is better. \label{tab:eg}}
\end{table}

When placing figures in \LaTeX, it's almost always best to use \verb+\includegraphics+, and to specify the  figure width as a multiple of the line width as in the example below
{\small\begin{verbatim}
   \usepackage[dvips]{graphicx} ...
   \includegraphics[width=0.8\linewidth]
                   {myfile.eps}
\end{verbatim}
}

\begin{figure}[t]
\begin{center}
\fbox{\rule{0pt}{2in} \rule{0.9\linewidth}{0pt}}
   %\includegraphics[width=0.8\linewidth]{egfigure.eps}
\end{center}
   \caption{Example of caption.  It is set in Roman so that mathematics (always set in Roman: $B \sin A = A \sin B$) may be included without an ugly clash.}
\label{fig:long}
\label{fig:onecol}
\end{figure}

\begin{figure}[t]
\begin{center}
\fbox{\rule{0pt}{2in} \rule{.5\linewidth}{0pt}}
\end{center}
   \caption{Example of a short caption, which should be centered.}
\label{fig:short}
\end{figure}

Figure and table captions should be 9-point Roman type as in Figures~\ref{fig:onecol} and~\ref{fig:short} and Table~\ref{tab:eg} and dynamically referenced with \verb'\'label{fig:$<$figure name$>$} or \verb'\'label{tab:$<$table name$>$} in the figure or table environment and \verb'Figure~\ref{fig:<figure name>}' or \verb'Table~\ref{tab:<table name>}'.  Replace \verb'<figure name>' or \verb'<table name>' with a short, descriptive name (which only used for referencing them in these commands).  Look at these example references to Figures~\ref{fig:onecol} and~\ref{fig:short} and Table~\ref{tab:eg} for example. You can also use this for referring to equations (such as in Equations~\ref{eq:1} and~\ref{eq:2}) or a section such as this reference to Section~\ref{sec:mat}.

%-------------------------------------------------------------------------
\subsection{Mathematics}\label{sec:mat}

Please number all of your sections and displayed equations.  It is
important for readers to be able to refer to any particular equation.  Just
because you didn't refer to it in the text doesn't mean some future reader
might not need to refer to it.  It is cumbersome to have to use
circumlocutions like ``the equation second from the top of page 3 column
1''.  (Note that the ruler will not be present in the final copy, so is not
an alternative to equation numbers).  All authors will benefit from reading
Mermin's description of how to write mathematics:
\url{http://www.pamitc.org/documents/mermin.pdf}.

Some example equations are shown in Equations~\ref{eq:1} and~\ref{eq:2}.
\begin{equation} \label{eq:1}
E(\mathbf{w}) = \frac{1}{2} \sum_{n=1}^N \left\lbrace y(x_n,\mathbf{w})-t_n\right\rbrace^2
\end{equation}
Equation~\ref{eq:2} shows how to align multiple equations together (which would be useful for your derivations).
\begin{align} \label{eq:2}
E(\mathbf{w}) =& \frac{1}{2} \sum_{n=1}^N \left\lbrace y(x_n,\mathbf{w})-t_n\right\rbrace^2 \\
              =& \frac{1}{2} \sum_{n=1}^N \left\lbrace y(x_n,\mathbf{w})-t_n\right\rbrace^2 + \frac{\lambda}{2} \left\lVert\mathbf{w}\right\rVert^2
\end{align}

\begin{table}\centering
\begin{tabular}{ll}
 \verb'$conf_a$' &  $conf_a$ \\
 \verb'$\mathit{conf}_a$' & $\mathit{conf}_a$
\end{tabular}
\caption{Two examples of ways to write math equations.  See The \TeX book, p165. \label{tab:math}}
\end{table}


% Some sample sections
%-------------------------------------------------------------------------
\section{Approach}
\subsection{Data}
Describe all the data you used using exact amounts.  For later projects, how you extracted the features out of the data. 
\subsection{Methods}
Describe the methods you used in your project.  For example, any derivations or proofs you have used.
\subsubsection{Regression using error minimization }
An example \verb'\subsubsection{}' to organize your multiple methods.  
\paragraph{Example Paragraph:} You may also use \verb'\paragraph{}' if you don't want it numbered and the text to start on the same line.
\section{Results}
You detailed results including figures and tables for each of your experiments.  Make sure to refer to each figure and table in your text.
\section{Conclusion}
Your take away from the project.  Any major issues you had while accomplishing the project.  Maybe add suggestions for any improvement to the project.
{\small
\bibliographystyle{IEEEtranN}
\bibliography{egbib}}
\end{document}